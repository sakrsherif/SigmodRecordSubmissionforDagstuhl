\subsection{Stream Processing Languages}

This tutorial gave an overview of several styles of stream processing
languages. The tutorial illustrated each style (e.g., relational,
synchronous, etc.) with a representative example language. Of course,
for each style, there is an entire family of languages, and this
tutorial did not aim to be exhaustive.

A \emph{stream} is a conceptually infinite ordered sequence of data
items, and a \emph{streaming application} is a computer program that
continuously ingests input streams and produces output streams.  A
\emph{stream processing language} is a DSL (domain-specific language)
for writing streaming applications. Some stream processing languages
are DSELs (domain-specific embedded languages~\cite{hudak_1998}),
meaning they are advanced library hosted by a general-purpose
language. But for clarity, this tutorial focused on stand-alone (not
embedded) DSLs.

\emph{Streaming SQL} dialects are an attempt to be for streaming data
what SQL has been for data stored in a database. A prominent example
is CQL~\cite{arasu_babu_widom_2006}. CQL extends the familiar
select-from-where syntax of SQL with windows that turn the recent
history of a stream into a relation, as well as with constructs for
watch the changes happening to a relation over time and deriving a
stream from them. CQL is implemented via translation into a streaming
extension of relational algebra.

\emph{Synchronous Dataflow} languages offer streaming with
deterministic concurrency for reliable embedded control systems. An
example is StreamIt~\cite{thies_et_al_2002}, where streaming
applications are graphs composed of only four constructs: individual
operators, pipe\-lines of operators, feedback loops, and split-merge
topologies that implement task parallelism. The StreamIt compiler
determines a repeating steady-state schedule, then exploits that for
optimizations such as operator fusion and data parallelism.

\emph{Big-Data Streaming} languages focus on large-scale stream
processing applications with a variety of data and processing
requirements. An example is SPL~\cite{hirzel_schneider_gedik_2017}.
SPL offers parallelism across both multi-core machines and
multi-machine clusters. SPL addresses a variety of data requirements
with rich data types and assorted parsing operators. And SPL addresses
a variety of processing requirements by letting programmers write new
first-class streaming operators in their language of choice.

\emph{Complex Event} patterns describe how to detect higher-level
events from a sequence of low-level events in a stream. In general
computing, the most widely used formalism for patterns over sequences
is regular expressions. Therefore, MatchRegex uses regular expressions
for patterns over streams~\cite{hirzel_2012}. Match\-Regex uses simple
predicates over stream data items to detect low-level events. In
addition, it supports a variety of aggregation functions over events
to both guide and summarize matches.

\begin{alltt}TODO\scriptsize 0.75/1 pages
- Reactive: ActiveSheets \cite{vaziri_et_al_2014}
- Controlled Natural Language: META \cite{arnold_et_al_2016}
\end{alltt}

Overall, the field of stream processing languages is diverse.  Efforts
to consolidate and standardize should be informed by an overview of
the state of the art, which this tutorial provided. The easier
streaming languages are to use, the more they contribute to the
democratization of streaming.
