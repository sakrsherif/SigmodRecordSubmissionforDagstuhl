\subsection{Stream Processing Languages}

This tutorial gave an overview of several styles of stream processing
languages. The tutorial illustrated each style (e.g., relational,
synchronous, etc.) with a representative example language. Of course,
for each style, there is an entire family of languages, and this
tutorial did not aim to be exhaustive.

\begin{alltt}TODO\scriptsize 0.75/1 pages
- DSELs: \cite{hudak_1998}
- definitions:
  - A stream is a conceptually infinite ordered sequence
    of data items.
  - A streaming application is a computer program that continuously
    ingests input streams and produces output streams.
  - A stream processing language is a DSL (domain-specific language)
    for writing streaming applications.
- Streaming SQL: CQL \cite{arasu_babu_widom_2006}
- Synchronous Dataflow: StreamIt \cite{thies_et_al_2002}
- Explicit stream graph: SPL \cite{hirzel_schneider_gedik_2017}
- Complex Events: MatchRegex \cite{hirzel_2012}
- Reactive: ActiveSheets \cite{vaziri_et_al_2014}
- Controlled Natural Language: META \cite{arnold_et_al_2016}
\end{alltt}

Overall, the field of stream processing languages is diverse.  Efforts
to consolidate and standardize should be informed by an overview of
the state of the art, which this tutorial provided. The easier
streaming languages are to use, the more they contribute to the
democratization of streaming.
