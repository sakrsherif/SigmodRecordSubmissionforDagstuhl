\subsection{Languages and Abstractions}\label{sec:wg_lang}

\begin{alltt}TODO\scriptsize 1 page
- introductory paragraph
  - working group focused on identifying challenges
  - see Section~\ref{sec:tut_lang} for definition and survey
- Challenge: Variety of data models (~0.25 pages)
  - definitions
    - data model organizes elements of data with respect to their
      semantics, their logical composition into data structures,
      and their physical representation
  - why is it important
    - variety of data models driven by the producers and consumers
      of streaming application, as well as internal conversion and
      processing needs
  - why difficult
    - little consensus among streaming languages:
      StreamIt single number~\cite{thies_et_al_2002},
      CQL tuple of numbers~\cite{arasu_babu_widom_2006},
      C-SPARQL entire self-describing graph~\cite{barbieri_et_al_2009}
  - what's unique to streaming
    - data model affects speed of serialization, transmission,
      compression, as well as dynamic checks for the presence
      or absence of certain fields
    - in online setting, no time for separate batch data integration
  - what's unique to languages
    - for some languages, the design revolves primarily around
      the data model, e.g., relational algebra on tuples, or
      path expressions on XML trees or RDF graphs
    - compiler opportunity for error messages and optimizations
  - goal(s)
    - language supports different data models
    - programmer uses logical data model that is easiest for them
    - compiler and runtime system choose best physical
      representation and transform program accordingly
  - metrics
    - expressive power, like relational completeness
    - performance (throughput, latency, resource consumption)
- Challenge: Veracity with simplicity (~0.25 pages)
  - definitions
    - conformance to the facts, accuracy
  - why is it important
    - sensor limitations (precision, energy, memory)
    - long-running and loosely coupled, sources come and go
    - approximate algorithms~\cite{babcock_et_al_2002}, stream mining~\cite{gaber_zaslavsky_krishnaswamy_2005}
  - why difficult
    - lack of ground truth or gold standard
    - hard to anticipate and test every eventuality
  - what's unique to streaming
    - real-time response without having seen all the data
    - due to online setting, no time for separate batch data cleansing
    - in distributed setting, no global clock~\cite{lamport_1978}
  - what's unique to languages
    - studies in pervasive handling of uncertainty in
      stream-relational algebra \cite{ali_et_al_2009} \cite{tran_et_al_2010}
    - can limit operators to support revision and retraction
    - probabilistic programming languages can track uncertainty~\cite{gordon_et_al_2014}
  - goal(s)
    - minimize compounding uncertainty
    - quality-aware and adaptive
    - keep language simple and expressive
  - metrics
    - precision, recall, f-score, Pareto optimality
    - optimizing vs. satisficing
- Challenge: Adoption (~0.25 pages)
  - definitions
    - adoption in the sense of broad acceptance and use
  - why is it important
    - adoption of a streaming language would drive adoption of
      streaming technologies in general
    - more attractive for student to invest time to learn
    - broader pool of skilled people to hire for company
    - drive libraries, tools, benchmarks, optimizations, etc.
  - why difficult
    - large variety of approaches, none is widely adopted
  - what's unique to streaming
    - domain is young, fast-moving, prone to vendor lock-in
  - what's unique to languages
    - no consensus on which features are the most important
      and which can be omitted to reduce complexity
    - need well-defined semantics (counter DSEL trend)
  - goal(s)
    - community agrees upon few languages that get widely adopted
  - metrics
    - number of users: job postings, resumes, courses, LoC,
      citations in papers, mentions in support forums, ...
    - number of systems that support it, industry standard
- closing paragraph
  - we hope this summary of the working group discussion helps
    guide future research in novel and impactful directions
\end{alltt}
