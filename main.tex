\documentclass[preprint]{sig-alternate-10pt}

\usepackage{alltt}
\usepackage{balance}
\usepackage[hyphens]{url}
\def\UrlFont{\scriptsize}
%\newcommand*{\url}[1]{\textsf{\scriptsize #1}}
\usepackage[para]{footmisc}

\usepackage{etoolbox}
\usepackage{graphicx}
\usepackage[font=small,labelfont=bf]{caption}
\apptocmd{\thebibliography}{\scriptsize}{}{}
\usepackage[utf8]{inputenc}

\usepackage[paperwidth=8.5in,paperheight=11in,margin=1in]{geometry}

\newcommand{\para}[1]{\vspace{2mm}\noindent\textbf{#1}}
\newcommand{\name}[1]{\textsf{\small #1}}

\begin{document}

\title{Dagstuhl Seminar on Big Stream Processing}

\iffalse
\numberofauthors{5}
\newcommand*{\emailn}[1]{\textsf{\normalsize #1}}

\author{
\alignauthor
Sherif Sakr\\
  \affaddr{University of Tartu, Estonia}
  \emailn{sherif.sakr@ut.ee}
\alignauthor
Tilmann Rabl\\
  \affaddr{TU Berlin, Germany}\\
  \emailn{rabl@tu-berlin.de}
\alignauthor
Martin Hirzel\\
  \affaddr{IBM Research AI, USA}\\
  \emailn{hirzel@us.ibm.com}
\and
\alignauthor
Paris Carbone\\
  \affaddr{KTH EECS, Sweden}\\
  \emailn{parisc@kth.se}
\and
\alignauthor
Martin Strohbach\\
  \affaddr{AGT International, Germany}\\
  \emailn{mstrohbach@agtinternational.com}}
\fi

\numberofauthors{1}
\author{\sffamily\hspace*{-4.2mm}\begin{tabular}{c@{\hspace*{2mm}}c@{\hspace*{2mm}}c@{\hspace*{2mm}}c@{\hspace*{2mm}}c}
  \large Sherif Sakr
& \large Tilmann Rabl
& \large Martin Hirzel
& \large Paris Carbone
& \large Martin Strohbach\\
  \normalsize Uni.\ of Tartu, Estonia
& \normalsize TU Berlin, Germany
& \normalsize IBM Research
& \normalsize KTH EECS, Sweden
& \normalsize AGT International, Germany\\
  \small sherif.sakr@ut.ee
& \small rabl@tu-berlin.de
& \small hirzel@us.ibm.com
& \small parisc@kth.se
& \small mstrohbach@agtinternational.com
\end{tabular}}

\maketitle

\begin{abstract}
Stream processing can generate insights from big data in real time as it is
being produced. This paper reports findings from a 2017 seminar on big
stream processing, focusing on applications, systems, and languages.
\end{abstract}

\section{Overview}\label{sec:overview}

As the world gets more instrumented and connected, we are witnessing a
flood of digital data that is getting generated, at high velocity,
from different hardware (e.g., sensors) or software in the form of
streams of data. Examples of this phenomenon are crucial for several
applications and domains including financial markets, surveillance
systems, manufacturing, smart cities, and scalable monitoring
infrastructure. In these applications and domains, there is a crucial
requirement to collect, process, and analyze big streams of data in
order to extract valuable information, to discover new insights in
real-time, and to detect emerging patterns and outliers. Since 2011
alone, several systems (e.g.,
\textsf{Storm}~\cite{toshniwal_et_al_2014},
\textsf{Apache Apex}\footnote{\url{https://apex.apache.org/}},
\textsf{Spark Streaming}~\cite{zaharia_et_al_2013},
\textsf{Apache Flink}~\cite{carbone_et_al_2015}, and
\textsf{Heron}~\cite{kulkarni_et_al_2015}) have
been introduced to tackle the real-time processing of big streaming
data. However, there are several challenges and open problems that
need to be addressed in order to improve the state-of-the-art in this
domain and to achieve wide adoption of big stream processing systems
by large number of users and enterprises~\cite{sakr2016big}.

This report is based on a five-day workshop on ``\emph{Big Stream
  Processing Systems}'' that took place at Schloss Dagstuhl in Germany
from 29~October to 3~November 2017\footnote{\url{http://www.dagstuhl.de/en/program/calendar/semhp/?semnr=17441}}. The workshop was attended by 29
researchers from from 13 countries. Participants came from different
communities including systems, query languages, benchmarking, stream
mining and semantic stream processing. A major benefit of this
workshop was the opportunity for scholars from different communities
to get exposure to each other and get freely engaged in direct and
interactive discussions. The seminar program consisted of tutorials on
the main topics of the workshop, lightning talks by participants on
their research, and two working groups dedicated to deeply
investigating selected topics. The first group focused on applications
and system of big stream processing while the second group focused on
analyzing the state-of-the-art of stream processing languages. This
report presents highlights from the workshop program.

\section{Tutorials}
The tutorials of the seminar aimed at sharing knowledge between
attendees from different communities, offering perspectives for group
discussions.

\subsection{IoT Stream Processing Applications}
\begin{alltt}TODO\scriptsize 0.75/1 pages
\end{alltt}
As part of the application tutorial we have picked a specific class of streaming applications, i.e. IoT applications that are concerned with interpreting and conceptualizing sensor data in real-time. Here we describe applications in two distinct domains. The first domain relates to Sports and Entertainment in which quantifiable insights about sports and entertainment events are created from sensors deployed at a venue. The second domain relates to Industry 4.0 in which sensor data from production machines is used to reduce energy costs and operations and maintenance costs. 

The application examples are based on commercial deployments that run on top of AGT International's \footnote{\url{http://www.agtinternational.com}} Internet of Things Analytics (IoTA) platform. IoTA is an IoT-based AI platform that provides cognitive and emotional computing skills to understand complex physical environments in real-time.

We use the two domains for two purposes: (1) With the applications described for sports and entertainment we illustrate the specific characteristics of IoT streaming applications and the associated challenge of choosing an appropriate streaming infrastructure. (2) We use the applications for Industry 4.0 to illustrate how stream processing applications can be benchmarked using the HOBBIT benchmarking platform.

\subsubsection{Sports and Entertainment}
Applications we consider in the sports and entertainment domain provide real-time narratives about key events that are happening at such an event. This way it is not necessary to watch the whole event, but one can be notified in real-time about highlights at the event based on insights created form the sensor data. For instance AGT International provides such analytic skills for the Basketball Euroleague\footnote{\url{http://www.euroleague.net/}} teams. This is achieved by using smart shirts worn by players, microphones deployed to monitor the audience, and using cameras and wristbands. These sensors in combination with play-by-play data allows IoTA to recognize behaviour, emotions, activities, actions, pressure and other physical aspects about a game. These insights are related to players, teams, fans and family and provided as semantic data streams that can for instance be used for distribution via social media. A more technical overview of the IoTA platform and based on this application is available online\footnote{\url{https://www.portal.reinvent.awsevents.com/connect/sessionDetail.ww?SESSION_ID=14367}}
%Figure Basketball ?

Other applications in this domain include mixed martial art fights, professional bull riding and the detection of event highlights at mass sport events such as the Color Run. For mixed martial arts AGT uses, among others, sensors embedded in fighters glove \footnote{\url{http://mmajunkie.com/2017/12/ufc-219-artificial-intelligence-glove-sensors-approved-by-nsac}} based on which a range of insights including punch strength and fighter's stress level are derived. Here it is important that insights can be delivered in real-time without noticeable delay compared to a broadcast of the fight. The full analytical capabilities and application scenario has been described by AGT International's CEO Mati Kochavi as part of the keynote at this years AWS re:invent conference\footnote{\url{https://youtu.be/vataVq9gY_o}}

\begin{alltt}TODO:\scriptsize add UFC keynote picture: \url{https://pbs.twimg.com/media/DP6L8S9XkAYFMCw.jpg}
\end{alltt}

In professional bull riding, AGT uses sensors that are attached to riders and bulls in order to quantify the bull's and rider's performance\footnote{\url{https://www.businesswire.com/news/home/20170822006048/en/Heed-Accelerates-Growth-Ground-Breaking-Partnerships-Professional-Basketball}}. As this information is among others used for automatic scoring, it is of particular importance that analytic results are available as soon as the ride is finished. 

In a similar fashion we use a range of wearable sensors for creating event highlights for participants at mass sport events. In the CPaas.io project we have developed an application that uses action cameras and fitness bands to automatically detect event highlights based on runner's activity, emotions, dance energy levels, and many more metrics. In this application real-time aspects include scenarios in which event highlights are being directly sent to friends of the participants. 
%sent to friends of participatns

\subsubsection{Industry 4.0}
\subsubsection{Choosing the Streaming Infrastructure}
When choosing a streaming infrastructure fulfilling the requirements of the above application scenarios, we face the following challenges
\begin{itemize}
\item No standard query language for streaming applications 
\item Multitude of streaming technologies make it hard to identify appropriate technology
\item JVM Legacy: ineffcient for distributed real-time processing
\item Flexibility for Programming Languages
\item Low Latencies and shortlived stream processing pipelines
\item Framework efficiency
\item Semantic mapping
\end{itemize}
\subsubsection{Benchmarking IoT Streaming Applications}

\subsection{Big Stream Processing Systems} 
\begin{alltt}TODO\scriptsize 0.75/1 pages
\end{alltt}
\subsection{Stream Processing Languages}

This tutorial gave an overview of several styles of stream processing
languages. The tutorial illustrated each style (e.g., relational,
synchronous, etc.) with a representative example language. Of course,
for each style, there is an entire family of languages, and this
tutorial did not aim to be exhaustive.

A \emph{stream} is a conceptually infinite ordered sequence of data
items, and a \emph{streaming application} is a computer program that
continuously ingests input streams and produces output streams.  A
\emph{stream processing language} is a DSL (domain-specific language)
for writing streaming applications. Some stream processing languages
are DSELs (domain-specific embedded languages~\cite{hudak_1998}),
meaning they are advanced library hosted by a general-purpose
language. But for clarity, this tutorial focused on stand-alone (not
embedded) DSLs.

\emph{Streaming SQL} dialects are an attempt to be for streaming data
what SQL has been for data stored in a database. A prominent example
is CQL~\cite{arasu_babu_widom_2006}. CQL extends the familiar
select-from-where syntax of SQL with windows that turn the recent
history of a stream into a relation, as well as with constructs for
watch the changes happening to a relation over time and deriving a
stream from them. CQL is implemented via translation into a streaming
extension of relational algebra.

\emph{Synchronous Dataflow} languages offer streaming with
deterministic concurrency for reliable embedded control systems. An
example is StreamIt~\cite{thies_et_al_2002}, where streaming
applications are graphs composed of only four constructs: individual
operators, pipe\-lines of operators, feedback loops, and split-merge
topologies that implement task parallelism. The StreamIt compiler
determines a repeating steady-state schedule, then exploits that for
optimizations such as operator fusion and data parallelism.

\emph{Big-Data Streaming} languages focus on large-scale stream
processing applications with a variety of data and processing
requirements. An example is SPL~\cite{hirzel_schneider_gedik_2017}.
SPL offers parallelism across both multi-core machines and
multi-machine clusters. SPL addresses a variety of data requirements
with rich data types and assorted parsing operators. And SPL addresses
a variety of processing requirements by letting programmers write new
first-class streaming operators in their language of choice.

\emph{Complex Event} patterns describe how to detect higher-level
events from a sequence of low-level events in a stream. In general
computing, the most widely used formalism for patterns over sequences
is regular expressions. Therefore, MatchRegex uses regular expressions
for patterns over streams~\cite{hirzel_2012}. Match\-Regex uses simple
predicates over stream data items to detect low-level events. In
addition, it supports a variety of aggregation functions over events
to both guide and summarize matches.

\begin{alltt}TODO\scriptsize 0.75/1 pages
- Reactive: ActiveSheets \cite{hirzel_et_al_2016}
- Controlled Natural Language: META \cite{arnold_et_al_2016}
\end{alltt}

Overall, the field of stream processing languages is diverse.  Efforts
to consolidate and standardize should be informed by an overview of
the state of the art, which this tutorial provided. The easier
streaming languages are to use, the more they contribute to the
democratization of streaming.

\section{Working Groups}
During the seminar, two separate working groups formed to discuss
current challenges in streaming applications and systems and in
streaming languages.

\subsection{Applications and Systems}
\begin{alltt}TODO\scriptsize 1 page
\end{alltt}

\subsection{Languages and Abstractions}\label{sec:wg_lang}

Based on the definitions and survey from the corresponding tutorial
(see Section~\ref{sec:tut_lang}), this working group identified and
described three challenges faced by stream processing languages.

\emph{Variety of data models} is a challenge for stream processing
languages. A data model organizes elements of data with respect to
their semantics, their logical composition into data structures, and
their physical representation. Producers and consumers of streams to
and from a streaming application dictate data models it must handle,
and the application's own conversion and processing needs drive
additional data-model variety.  There is no consensus on what a stream
data item is. At one extreme, in StreamIt, each data item is a simple
number~\cite{thies_et_al_2002}, while at the other extreme, C-SPARQL
streams entire self-describing graphs~\cite{barbieri_et_al_2009}.
Streaming languages have so far failed to consolidate on a data model
because data-model variety is a difficult challenge.

Data-model variety causes streaming-specific issues, since the data
model affects the speed of serialization, transmission, compression,
and dynamic checks for the presence or absence of certain fields, and
because the online setting leaves no time for separate batch data
integration. Some stream processing languages are designed around
their data model, e.g., CQL on tuples~\cite{arasu_babu_widom_2006} or
path expressions on XML trees~\cite{diao_et_al_2002}. Furthermore, the
data model enables streaming-language compilers to provide helpful
error messages and optimizations.

The goal is for streaming languages to let the programmer use the
logical data model they find most convenient while letting the
compiler choose the best physical representation. Metrics of success
are the expressive power of the language along with its throughput,
latency, and resource consumption.

\emph{Veracity with simplicity} is a challenge for stream processing
languages. Veracity means producing accurate and factual results, and
simplicity means avoiding unnecessary language complexity. There are
several reasons why streaming veracity is hard. Sensors producing
input data have limitations on precision, energy, and memory. In
long-running and loosely-coupled streaming applications, stream
sources come and go. And approximate stream
algorithms~\cite{babcock_et_al_2002} and stream
mining~\cite{gaber_zaslavsky_krishnaswamy_2005} introduce additional
uncertainty. This is compounded by the lack of ground truth in an
online setting, and by the difficulty of anticipating and testing
every eventuality.

Veracity causes streaming-specific issues, since it requires accurate
real-time responses without having seen all the data, and because the
online setting leaves no time for separate batch data cleansing. Also,
streaming is often used in a distributed setting, where there can be
no global clock~\cite{lamport_1978}. Some streaming languages are
studies in handling uncertainty on top of stream-relational
algebra~\cite{ali_et_al_2009,tran_et_al_2010}, but restricting stream
operators to support retraction or uncertainty propagation limits
expressiveness and raises complexity. A more general solution may use
probabilistic programming to handle uncertainty in a principled
way~\cite{gordon_et_al_2014}.

The goal is for streaming languages to help minimize compounding
uncertainty by being quality-aware and adaptive while remaining
simple, expressive, and fast. This inherently leads to multiple
metrics (e.g., precision, recall, throughput, latency) and
harder-to-quantify objectives (simplicity, expressiveness). One can
maximize one set of metrics while satisfycing a threshold on the
others, or one can seek Pareto-optimal
solutions~\cite{zhang_hirzel_grove_2016}.

\emph{Adoption} is a challenge for stream processing languages: while
there are many of them, none have reached broad acceptance and use.
The community should care about adoption of streaming languages
because it would drive adoption of streaming technologies in general.
A widely-adopted language is more attractive for students to learn,
leading to a bigger pool of skilled people to hire for companies.
Furthermore, a widely-adopted streaming language would drive more
mature libraries, tools, benchmarks, optimizations, etc. The lack of a
dominant language indicates that adoption is a difficult goal.

\begin{alltt}TODO\scriptsize 1 page
- Challenge: Adoption (~0.25 pages)
  - what's unique to streaming
    - domain is young, fast-moving, prone to vendor lock-in
  - what's unique to languages
    - no consensus on which features are the most important
      and which can be omitted to reduce complexity
    - need well-defined semantics (counter DSEL trend)
  - goal(s)
    - community agrees upon few languages that get widely adopted
  - metrics
    - number of users: job postings, resumes, courses, LoC,
      citations in papers, mentions in support forums, ...
    - number of systems that support it, industry standard
- closing paragraph
  - we hope this summary of the working group discussion helps
    guide future research in novel and impactful directions
\end{alltt}

\section{Conclusion}\label{sec:conclusion}
The tutorials, presentations, dialogs, and working groups at the ``\emph{Big Stream
  Processing Systems}'' workshop provided an overview of current developments and emerging issues in the areas of: systems,
computational models, architectures and languages for processing large-scale streaming data.
In this report, we highlighted the main outcomes of the workshop. The discussions of the workshop have also revealed the existence of several open challenges and interesting future research directions including the focus on (1) semantic data access and reasoning, (2) defining a standardized query language for streaming applications, (3) providing better support for machine learning including a wide range of data science programming languages (Python, R, Julia), and (4) improving optimizations for low latencies and short-lived stream processing pipelines.

\section*{Acknowledgements}
The work presented in this paper has partly been funded by the H2020 projects HOBBIT and CPaas.io under the grant agreement numbers 688227 and 723076.
\bibliographystyle{abbrv}
\balance
\bibliography{biblio}

\end{document}
