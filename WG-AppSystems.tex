\subsection{Applications and Systems}

In this working group, participants discussed characteristics and open 
challenges of stream processing systems. The discussions mainly focused on 
the topics state management, transactions, and pushing computation to the edge.

\emph{State management} -- Modern streaming systems are stateful, which means they can remember the state 
of the stream to some extent. A simple example is a counting operator that 
counts the number of elements seen so far. While even simple state like this 
poses several challenges in streaming setups (such as fault tolerance and 
consistency), many use cases call for more advanced state management 
capabilities. An example is the combination of streaming and batch data. This
is for example required when combining the history of a user with her current
activity or when finding matching advertisement campaigns with current activity
a popular example of such a setup is modeled in the Yahoo! Streaming 
Benchmark~\cite{Chintapalli2016BenchmarkingSC}. Today, most setups deal with
such challenges by combining different systems (e.g., a key value store for
state and a streaming system for processing), however, it is desirable to have 
both in a single system for consistency and manageability reasons. 

State can be considered the equivalent of a table in a database system. As a 
result, besides the combination of stream and state, several high level
operations can be identified: conversion of streams to tables (e.g., storing 
a stream), conversion of tables to streams (e.g., scanning a table), as well 
operations only on tables or streams (joins, filters, etc.). The management of 
state opens the design space in between existing stream processing systems and 
database systems, which has only been partially explored by current systems.
In contrast to database systems, stream systems typically operate in a reactive
manner, i.e., they have no control over the incoming data stream, specifically, 
they do not control and define the consistency and order semantics in the 
stream. This requires advanced notions of time and order as for example 
specified for streams in the dataflow model \cite{43864}, for state and stream
this remains an open field of research. 


\emph{Transactions}

A further discussion topic where transactions in stream processing systems. 
The main difference between traditional database transactions and stream 
processing transactions is that in databases the computation moves and 
data stays (in the system), in stream processing systems the computation stays
and the data moves to the computation (and out again). 

Considering state management, the form of transactions as applied in databases
can also be used in a stream processing system, if the state is managed in a
transactional way. However, the operations on streams themselves can be 
transactional and then we can differentiate between single tuple transactions 
and multiple tuple transactions. For multiple tuple transactions, the 
transaction can only commit when all tuples are consumed. The tuples then have
to pass the whole operator graph or at least the transactional subgraph. 

The semantics of transactions on streams is currently still an open 
field of research.

By \emph{pushing computation to the edge} of a network, stream processing can 
be highly distributed and decentralized. This is very useful when preprocessing
or filtering can be done without a centralized view of the data. Especially, in 
setups with high communication cost or slow connections (e.g., mobile 
connections), it makes sense to not send all data to a central server, but 
distribute the computation. A logical first step is filtering, but aggregations
and even more complex operations can be pushed to the edge, if possible. Many
modern scenarios prohibit centralized data storage, which further encourages
distributed setups with early aggregations. Primary points of research are the
declarativity for specifying highly distributed data processing programs and 
the architecture of systems to support these use cases. 


\emph{Other topics discussed} were ad hoc queries and graph stream processing.
Most current systems only discuss long running queries, but in many use cases 
(e.g., sports, automotive) streams can be short lived as can be stream queries. 
